\documentclass{article}
\usepackage[a4paper, total={6in, 8in}]{geometry}
\usepackage{amsthm}
\usepackage{amssymb}
\usepackage{amsfonts}
\usepackage{amsmath}
\usepackage{setspace}
\usepackage{xcolor}
\usepackage{cite}
\usepackage[hidelinks]{hyperref}
\usepackage{xurl}
\usepackage{microtype}
\usepackage{enumitem}

% macro for bra kets
\usepackage{mathtools}
\DeclarePairedDelimiter\bra{\langle}{\rvert}
\DeclarePairedDelimiter\ket{\lvert}{\rangle}
\DeclarePairedDelimiterX\braket[2]{\langle}{\rangle}{#1\,\delimsize\vert\,\mathopen{}#2}

\newcommand{\R}{\mathbb{R}}
\newcommand{\C}{\mathbb{C}}
\newcommand{\eps}{\varepsilon}
\renewcommand{\Im}{\mathrm{Im}}
\renewcommand{\Re}{\mathrm{Re}}

\newcommand{\bb}{\mathbb}

\newcommand{\SL}{\mathrm{SL}}
\DeclareMathOperator{\diag}{\mathrm{diag}}
\renewcommand{\d}{\mathrm{d}}

\newtheorem{lemma}{Lemma}
\theoremstyle{remark}
\newtheorem{questionpart}{}
\renewcommand*{\thequestionpart}{\textbf{(\alph{questionpart})}}

\title{Lie Groups - Assignment 3}
\author{Quinten Vergeylen}
\date{}

\begin{document}
\setstretch{1.1}
\maketitle

\section*{Week 8, Exercise 1}


\section*{Week 9, Exercise 3}
\begin{enumerate}[label=(\roman*)]
    \item We assume $C(G)$ is equipped with the supremum norm $\lVert \, \cdot \, \rVert_\infty$. For any $g \in G$, the map $L_g : V \to V$ is linear. Since $V$ is a finite-dimensional normed space, $L_g$ is continuous. Now fix $\eps$ and select $(h, \psi) \in G \times V$. We want to show that there exists an open neighbourhood of $(h, \psi)$ such that $\lVert L_g(\phi) - L_h(\psi) \rVert_\infty < \eps$ for all $(g, \phi)$ in that neighbourhood.

    Let $B(\psi, \eps/2) \subset V$ be the open ball around $\psi$ with radius $\eps/2$ with respect to the supremum norm. Let $A = L_h^{-1}(B(\psi, \eps/2))$, which is open since $L_h$ is continuous. Then, for all $\phi \in A$, and for any $g \in G$,
    \begin{align*}
        \lVert L_g(\phi) - L_g(\psi) \rVert_\infty &= \sup_{x \in G} \lvert \phi(g^{-1}x) - \psi(g^{-1}x) \rvert \\
        &= \sup_{x \in G} \lvert \phi(h^{-1}x) - \psi(h^{-1}x) \rvert \\
        &= \lVert L_h(\phi) - L_h(\psi) \rVert_\infty \\
        &< \eps/2,
    \end{align*}
    where we used that the left multiplications $\ell_g$ and $\ell_{h^{-1}}$ are bijections. Furthermore, for any $g \in G$,
    \begin{equation*}
        \lVert L_g(\psi) - L_h(\psi) \rVert_\infty = \sup_{x \in G} \lvert \psi(g^{-1}x) - \psi(h^{-1}x) \rvert = \sup_{x \in G} \lvert \psi(g^{-1}hx) - \psi(x) \rvert.
    \end{equation*}
    %Since the domain of $\psi$, namely $G$, is compact, and $\psi$ is continuous, we can apply uniform continuity on topological groups (similarly to the proof of Proposition 20.10): there exists an open neighbourhood $U \subset G$ of $e$ such that $\lvert \psi(y^{-1}x) - \psi(x) \rvert < \eps/3$ for all $x, y \in U$. Therefore, $\lVert L_g(\psi) - L_h(\psi) \rVert_\infty < \eps/2$ for all $g \in hU$.
    By (ii) (which we can use since it does not depend on the result we are proving here), there is a neighbourhood $U$ of $e \in G$ such that the latter expression is less than $\eps/2$ for all $g \in hV$, where $V$ is the open set $\{x \in G \mid x^{-1} \in U\}$. (It is open since $x \mapsto x^{-1}$ is a homeomorphism of the topological group $G$.)

    Now $hV \times A$ is an open neighbourhood of $(h, \psi)$ such that, for all $(g, \phi) \in hV \times A$,
    \begin{equation*}
        \lVert L_g(\phi) - L_h(\psi) \rVert_\infty \leq \lVert L_g(\phi) - L_g(\psi) \rVert_\infty + \lVert L_g(\psi) - L_h(\psi) \rVert_\infty < \eps.
    \end{equation*}
    Since we chose $\eps >0$ and $(h, \psi) \in G \times V$ arbitrarily, it follows that $L : G \times V \to V$ is continuous. Clearly, it is also a still a left action, so $L$ is a continuous representation of $G$ in $V$.

    \item We will use the following Lemma.
    \begin{lemma} \label{lemma1}
        Let $G$ be a topological group. For any open neighbourhood $E \subset G$ of $e$, there exists an open neighbourhood $A$ of $e$ contained in $E$ such that $AA^{-1} \subset E$.
    \end{lemma}
    \begin{proof}
        The map
        \begin{equation*}
            \varphi : G \times G : (x, y) \mapsto xy^{-1}
        \end{equation*}
        is continuous since the group multiplication and inverse operation are continuous. Therefore, $\varphi^{-1}(E)$ is open. This set is nonempty since it contains $(e,e)$. Moreover, the projections $p_1(x, y) = x$ and $p_2(x, y) = y$ are open maps, so
        \begin{equation*}
            A := p_1(\varphi^{-1}(E)) \cap p_2(\varphi^{-1}(E))
        \end{equation*}
        is open, and nonempty since it contains $e$. Now for all $x, y \in A$, $xy^{-1} = \varphi(x, y) \in E$, so $e \in AA^{-1} \subset E$.
    \end{proof}

    Let $E_g = f^{-1}(B(f(g), \eps/2)) \, g^{-1}$ for every $g \in G$, where $B(f(g), \eps/2) \subset \bb{R}$ is the open interval centered at $f(g)$ with width $\eps$. For every $g \in G$, $e \in E_g$, so by Lemma \ref{lemma1} there is an open neighbourhood $A_g$ of $e$ contained in $E_g$ such that $A_gA_g^{-1} \subset E$. Let $B_g = A_g \cap A_g^{-1}$ for every $g \in G$. Then $g \in B_g g$ since $e \in B_g$, so $\{B_g g\}_{g \in G}$ is an open cover of $G$. Since $G$ is compact, there is a finite subcover $B_{g_1} g_1, \ldots, B_{g_r} g_r$. Let $U = B_{g_1} \cap \cdots \cap B_{g_r}$, an open neighbourhood of $e$.

    Now pick $u \in U$ and $g \in G$. There is an $i \in \{1, \ldots, r\}$ for which $g \in B_{g_i} g_i$. But $B_{g_i} \subset A_{g_i} \subset E_{g_i}$, so $g \in E_{g_i} g_i$. Moreover, $ug \in B_i B_i g_i \subset A_{g_i} A_{g_i}^{-1} g_i \subset E_{g_i} g_i$. Therefore,
    \begin{equation*}
        \lvert f(ug) - f(g) \rvert \leq \lvert f(ug) - f(g_i)\rvert + \lvert f(g_i) - f(g) \rvert < \frac{\eps}{2} + \frac{\eps}{2} = \eps.
    \end{equation*}

    \item Fix $g \in G$. Clearly, $L_g$ is linear. Since
    \begin{equation*}
        \lVert L_g(f) \rVert_\infty = \sup_{x \in G} \lvert L_g(f)(x) \rvert = \sup_{x \in G} \lvert f(g^{-1}x) \rvert = \sup_{x \in G} \lvert f(x) \rvert = \lVert f \rVert_\infty
    \end{equation*}
    (where we used that $\{g^{-1}x \mid x \in G\} = G$), $L_g$ is bounded linear operator between normed spaces, so it is continuous.

    Now we could proceed as in (i), since the second part of that proof does not depend on the finite dimensionality of the function space. However, here I will assume $G$ is a Lie group, so we can use Proposition 20.8 from van de Ban's lecture notes.
    
    Fix $f \in C(G)$ and select $\eps >0$. By part (ii), there exists a neighbourhood $V$ of $e \in G$ such that
    \begin{equation*}
        \lvert L_g(f)(x) - L_e(f)(x) \rvert = \lvert f(g^{-1}x) - f(x) \rvert < \eps
    \end{equation*}
    for all $x \in G$, $g \in V$. Indeed, we can take $V= U^{-1}$, where $U$ is the neighbourhood of $e$ as in (ii). The set $V$ is open since $V = \iota^{-1}(U)$ where $\iota$ is the inverse operation on $G$, which is continuous, and $e \in V$ since $e^{-1} = e$. It follows that,
    \begin{equation*}
        \lVert L_g(f)- L_e(f) \rVert_\infty = \sup_{x \in G} \lvert L_g(f)(x) - L_e(f)(x) \rvert \leq \eps
    \end{equation*}
    for all $g \in V$. Since $\eps>0$ was arbitrary, we have found that $g \mapsto L_g(f)$ is continuous at $e$.  By Proposition 20.8, $L : G \times C(G) \to C(G)$ is continuous, so the representation $L$ of $G$ on $C(G)$ is continuous.    
\end{enumerate}

\section*{Week 10, Exercise 3}
\begin{enumerate}[label = (\roman*)]
    \item Let $g \in G$. Then, for all $h \in G$,
    \begin{equation*}
        \pi(g)\pi(h)v = \pi(g, \pi(h, v)) = \pi(gh, v) = \pi(hg, v) = \pi(h, \pi(g, v)) = \pi(h)\pi(g)(v).
    \end{equation*}
    In the first and last equality we used the definition of $\pi$; in the second and fourth equality we used that $\pi$ is a group action; and in the third equality we used that $G$ is commutative. It follows that $\pi(g) : V \to V$ is an equivariant map. Then, by Shur's lemma, $\pi(g) = \lambda_g I_V$ for some $\lambda_g \in \bb{C}$. Therefore $\pi(g)$ leaves every subspace of $V$ invariant. This holds for every $g \in G$. Hence, if $\pi$ is irreducible, then $\dim V = 1$, since in that case the only linear subspaces of $V$ are $0$ and $V$ itself. Conversely, if $\dim V = 1$, then the only linear subspaces of $V$ are $0$ and $V$ itself, so $\pi$ is trivially irreducible.

    \item This is an immediate consequence of (i) and Corollary 20.19 (van de Ban). From the proofs of Corollary 20.19 and Lemma 20.18 it is clear that the subspaces are mutually orthogonal.
    
    \item The natural representation of $\mathrm{SO}(2)$ on $\bb{C}^2$ is unitary. Indeed, an element of $\mathrm{SO}(2)$ is represented by
    \begin{equation*}
        \begin{pmatrix}
        \cos \theta & \sin \theta \\ -\sin \theta & \cos\theta
        \end{pmatrix},
    \end{equation*}
    for some $\theta \in [0, 2\pi)$, which is a unitary matrix. Furthermore, the representation is finite-dimensional since $\bb{C}^2$. Lastly, $\mathrm{SO}(2)$ is commutative, since it is isomorphic to the clearly commutative group $S^1$.
    Therefore, we can use (i): $\dim \bb{C} > 1$, so the representation is not irreducible.

    \item 
\end{enumerate}

% \section*{Week 10, Exercise 4}
% \begin{enumerate}[label = (\roman*)]
%     \item It suffices to show that there is a $g\in \mathrm{SO}(n)$ such that $gv_j = e_j$ for$j=1,2$, where $e_j$ are the standard basis vectors in $\bb{C}_n$. Indeed, if $gv_j=e_j$ and $hw_j = e_j$ for $g, h \in \mathrm{SO}(n)$ and all $j$, then $h^{-1}g \in \mathrm{SO}(n)$ and $h^{-1}gv_j = w_j$ for all $j$.
    
%     By Example 15.6 (van de Ban),\footnote{Example 15.6 says $n \geq 0$ must hold, but I believe this to be a mistake since $\mathrm{SO(1)} = \{1\}$ does not act transitively on $S^0 = \{-1,1\} \subset \bb{R}$. However, the statements from the example do hold for $n \geq 1$. If the example did hold for $n=0$, then the argument I give in this exercise to show that $\pi$ is irreducible would contradict the result of exercise 3 of week 10.} $\mathrm{SO}(n)$ acts transitively on $S^{n-1} \subset \bb{R}^n$. Since $v_1$ and $e_1$ are both elements of $S^{n-1}$, there exists a $g_1 \in \mathrm{SO}(n)$ such that $g_1v_1 = e_1$. Now, $v_2$ is orthogonal to $v_1$, so, because $g_1$ preserves the inner product, $g_1v_2$ is orthogonal to $g_1v_1 = e_1$. Therefore $g_1v_2 \in \{(x_1, \ldots, x_n) \in S^{n-1} \mid x_1 = 0\}$, which we identify with $S^{n-2}$. Note also that $e_2 \in S^{n-2}$. As observed in Example 15.6, the stabiliser subgroup $\mathrm{SO}(n)_{e_1}$ consists of matrices of the form
%     \begin{equation*}
%         \begin{pmatrix}
%             1 & 0 \\ 0 & B
%         \end{pmatrix}
%     \end{equation*}
%     with $B \in SO(n-1)$, and the action of this subgroup on an element $\begin{pmatrix} 0 & w \end{pmatrix}^\top \in S^{n-2}$ is
%     \begin{equation*}
%         \begin{pmatrix}
%             1 & 0 \\ 0 & B
%         \end{pmatrix}
%         \begin{pmatrix} 0 \\ w \end{pmatrix}
%         =
%         \begin{pmatrix}
%             0 \\ Bw
%         \end{pmatrix},
%     \end{equation*}
%     so $\mathrm{SO}(n)_{e_1}$ acts exactly like $\mathrm{SO}(n-1)$ on $S^{n-2}$. This is again a transitive group action by Example 15.6 (since $n \geq 3$, so $n-1 \geq 1+1$, which is required in the example---see footnote 1 on this page), which implies that there is a $g_2 \in \mathrm{SO}(n)_{e_1} \subset \mathrm{SO}(n)$ such that $g_2(g_1v_2) = e_2$. Since $g_2$ is an element of the stabiliser subgroup at $e_1 = g_1v_1$, we have $g_2(g_1v_1) = e_1$. Therefore, the element $g = g_1g_2$ of $\mathrm{SO}(n)$ satisfies $gv_1 = e_1$ and $gv_2 = e_2$.

%     \item Let $T \in \mathfrak{c}$. Let $g \in \mathrm{SO(n)} \subset M_n(\bb{C})$ such that $gv_1 = w_1$ and $gv_2 = w_2$, as in (i). Then $Tg = gT$, so
%     \begin{equation*}
%         \langle Tw_1, w_2 \rangle = \langle Tgv_1, gv_2 \rangle = \langle gTv_1, gv_2 \rangle = \langle Tv_1, v_2 \rangle.
%     \end{equation*}
%     The last equality holds because elements of $\mathrm{SO}(n)$ preserve the standard inner product: for all $a, b \in \bb{C}^n$,
%     \begin{equation*}
%         \langle ga, gb \rangle = \langle a, \overline{g}^\top gb \rangle = \langle a, g^\top gb \rangle = \langle a, b \rangle,
%     \end{equation*}
%     where we used that $g$ has real components and $g\top g$ is the identity matrix since $g$ is orthogonal.

%     \item 
% \end{enumerate}


\end{document}