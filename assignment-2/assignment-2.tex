\documentclass{article}
\usepackage[a4paper, total={6in, 8in}]{geometry}
\usepackage{amsthm}
\usepackage{amssymb}
\usepackage{amsfonts}
\usepackage{amsmath}
\usepackage{setspace}
\usepackage{xcolor}
\usepackage{cite}
\usepackage[hidelinks]{hyperref}
\usepackage{xurl}
\usepackage{microtype}
\usepackage{enumitem}
\usepackage{bbm}

% macro for bra kets
\usepackage{mathtools}
\DeclarePairedDelimiter\bra{\langle}{\rvert}
\DeclarePairedDelimiter\ket{\lvert}{\rangle}
\DeclarePairedDelimiterX\braket[2]{\langle}{\rangle}{#1\,\delimsize\vert\,\mathopen{}#2}

\newcommand{\R}{\mathbb{R}}
\newcommand{\C}{\mathbb{C}}
\newcommand{\eps}{\varepsilon}
\renewcommand{\Im}{\mathrm{Im}}
\renewcommand{\Re}{\mathrm{Re}}

\newcommand{\bb}{\mathbb}

\newcommand{\SL}{\mathrm{SL}}
\DeclareMathOperator{\diag}{\mathrm{diag}}
\renewcommand{\d}{\mathrm{d}}

\theoremstyle{remark}
\newtheorem{questionpart}{}
\renewcommand*{\thequestionpart}{\textbf{(\alph{questionpart})}}

\title{Lie Groups -- Assignment 2}
\author{Quinten Vergeylen}
\date{}

\begin{document}
\maketitle
\setstretch{1.1}

\section*{Week 4: Exercise 1} %==============================
\begin{enumerate}[label=(\roman*)]
    \item By Theorem 4.8, $G_e \cong \bb{T}^p \times \bb{R}^q$ for certain $p, q \in \bb{N}$. Therefore, any subgroup of $G_e$ is isomorphic to a group of the form $T_1 \times \cdots \times T_p \times R_1 \times \cdots \times R_q$, where $T_j$ is a subgroup of $\bb{T}$ for every $j$ and $R_k$ is a subgroup of $\bb{R}$ for every $k$. We claim that $\bb{T}^p \cong \bb{T}^p \times \{0\}^p \subset G_e$ is the unique maximal compact subgroup of $G_e$, where $\{0\}$ denotes the trivial subgroup of $\bb{R}$.
    
    It is clear that $\bb{T}^p$ is a compact subgroup of $G_e$. Suppose that $L$ is a subgroup of $G_e$ which contains $\bb{T}^p$. Then $L$ is isomorphic to
    \begin{equation*}
        \bb{T}^p \times R_1 \times \cdots \times R_q
    \end{equation*}
    for certain subgroups $R_k$ of $\bb{R}$. If $R_k$ is any other subgroup of $\bb{R}$ other than $\{0\}$ for any $k$, then $L$ is not compact, since $\{0\}$ is the only compact subgroup of $\bb{R}$. It follows that $\bb{T}^p$ is a maximal compact subgroup of $G_e$. Lastly, we can use the reasoning above to assert that any compact subgroup of $G_e$ is isomorphic to 
    \begin{equation*}
        T_1 \times T_p \times \{0\}^p \cong T_1 \times T_p \subset \bb{T}^p
    \end{equation*}
    for certain subgroups $T_j$ of $\bb{T}$. Therefore, $K = \bb{T}^p$ is the unique maximal compact subgroup of $G_e$.

    \item Let $A = \{ g \in G \mid \exists n \in \bb{Z}_{>0} : g^n \in K \}$. We claim that $A$ is the unique maximal compact subgroup of $G$.
    
    First we show that $A$ is a subgroup of $G$. Let $g, h \in A$ and suppose $g^n \in K$ and $h^m \in K$. Then $(gh)^{nm} = g^{nm}h^{nm} = (g^n)^m (h^m)^n \in K$, where we used that $G$ is abelian and that $K$ is a subgroup of $G$. It follows that $gh \in K$. Furthermore, $(g^{-1})^n = (g^n)^{-1} \in K$, so $g^{-1} \in A$. Therefore, $A$ is a subgroup of $G$.

    Next, we show that $A$ is compact. Let $G_1 = G_e, G_2,  \ldots, G_k$ be the connected components of $G$ with the property for each $j \in \{1, \ldots, k\}$ that there exists a $h_j \in G_j$ and an $n_j \in \bb{Z}_{>0}$ such that $h_j^{n_j} \in K$. In other words, $G_1, \ldots, G_k$ are the connected components of $G$ with nonempty intersection with $A$. We know that there are finitely many such $G_j$ since $G$ has finitely many connected components by assumption.
    
    Fix $j \in \{1, \ldots, j\}$ for now and note that the left translation $\ell_{h_j} : G_e \to G_j$ is a diffeomorphism. Indeed, $\ell_{h_j}(e) = h_j \in G_j$ and $\ell_{h_j}(G_e)$ is both open and closed since $G_e$ is both of these and $\ell_{h_j} : G \to G$ is a diffeomorphism. It follows that $\ell_{h_j}(G_e)$ must be the connected component of $G$ containing $h_j$, which is precisely $G_j$. Now pick any $x \in G_j \cap A$, supposing $x^m \in K$. We now have $\ell_{h_j}^{-1}(x) = h_j^{-1}x \in G_e \cong \bb{T}^p \times \bb{R}^q$ and $(h_j^{-1}x)^{mn_j} \in K \cong \bb{T}^p \times \{0\}^p$. Let $\phi : G_e \to \bb{T}^p \times \bb{R}^q$ be a Lie group isomorphism and $\pi : \bb{T}^p \times \bb{R}^q \to \bb{R}^q$ the projection. The fact that $(\pi \circ \phi)((h_j^{-1}x)^{mn_j}) = 0 \in \bb{R}^q$ implies that $(\pi \circ \phi)(h_j^{-1}x) = 0$, since the only real number that becomes zero when added to itself $mn_j$ times is zero. Therefore, $h_j^{-1}x \in K$, and hence $xK = h_jK$. In particular, $x \in h_j K$.

    It now follows that $A = h_1K \cup \cdots \cup h_kK$. Indeed, we have just shown that any element $x$ of $A$ is contained in $h_jK$ if $G_j$ is the connected component of $G$ containing $x$. Conversely, $(h_jg)^{n_j} = h_j^{n_j}g^{n_j} \in K$ for every $g \in K$, so $h_jK \subset A$ for every $j$. Since $A$ is a finite union of compact sets $h_jK$, we have shown that $A$ is compact.

    Suppose $B$ is a subgroup of $G$ and $A \subsetneq B$. Fix $x \in B\setminus A$ and $j \in \{1, \ldots, k\}$ for which $x \in G_j$. Because $x \not\in A$, we have $x \not\in h_jK$, and therefore $h_j^{-1}x \not\in K$. Then $v := (\pi \circ \phi)(h_j^{-1}x) \in \bb{R}^q \setminus \{0\}$. Consequently, $\left((\pi \circ \phi)((h_j^{-1}x)^{mn_j})\right)_{m\in \bb{Z}} = (mn_jv)_{m \in \bb{Z}}$ (using the fact that $\pi \circ \phi$ is a group homomorphism) is a sequence in $\bb{R}$ of distinct elements. In particular, $((h_j^{-1}x)^{mn_j})_{m \in \bb{Z}}$ is a sequence of distinct elements of $B$. (This is because $h_j \in A \subset B$ and $x \in B$ and $B$ is a group.) Now write
    \begin{equation*}
        L_m = \left\{av \,\middle|\, \left(m-\frac{3}{4}\right)n_j < a < \left(m+\frac{3}{4}\right)n_j\right\} \subset \mathrm{span}\{v\} \subset \bb{R}^q
    \end{equation*}
    for every $m \in \bb{Z}$. Then $\{L_m \times (\mathrm{span}\{v\})^\perp\}_{m \in \bb{Z}}$ is an open cover of $\bb{R}^q$. Let
    \begin{equation*}
        U_m = (\pi \circ \phi)^{-1}\left(L_m \times (\mathrm{span}\{v\})^\perp\right) \cup G_2 \cup \cdots \cup G_k
    \end{equation*}
    for every $m \in \bb{Z}$, so $\{U_m\}_{m\in \bb{Z}}$ is an open cover of $G$ and in particular of $B$, by the continuity of $\pi \circ \phi$. Moreover, by the injectivity of $\pi \circ \phi$, every $U_{\overline{m}}$ contains precisely one element of the sequence $((h_j^{-1}x)^{mn_j})_{m \in \bb{Z}}$, namely $(h_j^{-1}x)^{\overline{m}n_j}$, and it is the only open set in the cover that contains this element. Therefore, the open cover $\{U_m\}$ does not admit a finite subcover over $B$, since any such subcover would omit elements of that sequence. Hence, $B$ is not compact.

    \textcolor{red}{Uniqueness???}
    %In particular, $((h_j^{-1}x)^{mn_j})_{m \in \bb{N}}$ is a sequence of distinct elements of $B$. (This is because $h_j \in A \subset B$ and $x \in B$ and $B$ is a group.) 
    %Now define $U_m = G \setminus \{(h_j^{-1}x)^{mn_j}\}$ for every $m \in \bb{N}$. Then $\{U_m\}_m$ is an open cover of $B$. It is clear, however, that it admits no finite subcover of $B$, since such a subcover would necessarily omit elements of the sequence $((h_j^{-1}x)^{mn_j})_{m \in \bb{N}}$. Therefore, $B$ is not compact. We have shown that $A$ is a maximal compact subgroup of $G$.
    % Then $\{U_m\}_m$ is an open cover of $G$, and in particular of $B$. Suppose that there is a finite subcover $U_{m_1}, \ldots, U_{m_r}$ of $B$. Let $\overline{m} \in \bb{N} \setminus \{m_1, \ldots, m_r\}$. Then
    % \begin{equation*}
    %     (\pi \circ \phi)\left((h_j^{-1}x)^{\overline{m}n_j}\right) \not\in \bigcup_{l=1}^r (\pi \circ \phi)
    % \end{equation*}
\end{enumerate}


\section*{Week 5: Exercise 3} %==============================
\begin{enumerate}[label=(\roman*)]
    \item \underline{Associativity:} Let $(n,h), (n',h'), (n'',h'') \in N \times H$. Then
    \begin{equation*}
        (n,h)\left((n',h')(n'',h'')\right) = (n,h)(n'\alpha_{h'}(n''), h'h'') = (n\alpha_h(n'\alpha_{h'}(n'')), h(h'h'')).
    \end{equation*}
    Since $H$ is a group, $h(h'h'') = (hh')h''$. Furthermore,
    \begin{equation*}
        n\alpha_h(n'\alpha_{h'}(n'')) = n\alpha_h(n')\alpha_h(\alpha_{h'}(n'')) = n\alpha_h(n')\alpha_{hh'}(n''),
    \end{equation*}
    using in the first equality that $\alpha_h$ is a group homomorphism and in the second that $\alpha$ is a group action. It follows that
    \begin{align*}
        (n,h)\left((n',h')(n'',h'')\right) = (n\alpha_h(n')\alpha_{hh'}(n''), (hh')h'') &= (n\alpha_h(n'), hh')(n'',h'') \\
        &= \left((n,h)(n',h')\right)(n'',h''),
    \end{align*}
    so the product is associative.

    \underline{Identity:} Let $(n,h) \in N \times H$. Let $e_N$ denote the identity in $N$ and $e_H$ the identity in $H$. Note that
    \begin{equation*}
        (e_N, e_H)(n,h) = (e_N \alpha_{e_H}(n), e_Hh) = (n,h)
    \end{equation*}
    and
    \begin{equation*}
        (n,h)(e_N, e_H) = (n\alpha_h(e_N), he_H) = (ne_N, h) = (n,h).
    \end{equation*}
    In the second to last equality, we used that $\alpha_h$ is a group homomorphism. It follows that $(e_N, e_H)$ is an identity element in $N\times H$ for this product.

    \underline{Inverse:} Let $(n,h) \in N\times H$. Then
    \begin{align*}
        (n,h)(\alpha_{h^{-1}}(n^{-1}), h^{-1}) = (n\alpha_h(\alpha_{h^{-1}}(n^{-1})), hh^{-1}) = (n\alpha_{e_H}(n^{-1}), e_H) &= (nn^{-1}, e_H) \\
        &= (e_N, e_H),
    \end{align*}
    where in the second and third equalities we used that $\alpha$ is a group action, and
    \begin{align*}
        (\alpha_{h^{-1}}(n^{-1}), h^{-1})(n,h) = (\alpha_{h^{-1}}(n^{-1})\alpha_{h^{-1}}(n), h^{-1}h) = (\alpha_{h^{-1}}(n^{-1}n), e_H) = (e_N, e_H),
    \end{align*}
    where in the first and last equalities we used that $\alpha_{h^{-1}}$ is a group homomorphism. Therefore, $(\alpha_{h^{-1}}(n^{-1}), h^{-1})$ is the inverse of $(n,h)$ in $N\times H$ for this product.

    \item Let $(n,e_H) \in N\times \{e_H\}$. Let $(n',h') \in G$. Then
    \begin{align*}
        (n',h')(n,e_H)(n',h')^{-1} &= (n',h')(n,e_H)(\alpha_{h'^{-1}}(n'^{-1}), h'^{-1}) \\
        &= (n'\alpha_{h'}(n), h')(\alpha_{h'^{-1}}(n'^{-1}), h'^{-1}) \\
        &= (n'\alpha_{h'}(n)\alpha_{h'}(\alpha_{h'^{-1}}(n'^{-1})), e_H) \in N \times \{e_H\}.
    \end{align*}
    Since $(n,e_H)$ and $(n',h')$ were arbitrary, it follows that $N \cong N\times\{e_H\}$ is a normal subgroup of $G$.

    \item Since cartesian products of manifolds are manifolds, $G$ is a manifold. In this smooth structure, the projections $\pi_N : G = N \times H \to N$ and $\pi_H : G = N \times H \to H$ are smooth. Let $\mu_N$ and $\mu_H$ be the multiplications on $N$ and $H$ respectively, which are both smooth since $N$ and $H$ are Lie groups. The multiplication $\mu : G \times G \to G$ on $G$ is given by
    \begin{equation*}
        \mu(g_1, g_2) = (\mu_N(\pi_N(g_1), \alpha(\pi_H(g_1), \pi_N(g_2))), \mu_H(\pi_H(g_1), \pi_H(g_2))).
    \end{equation*}
    This is a composition of smooth maps, so $\mu$ is smooth. Furthermore, let $\iota_N$ and $\iota_H$ be the inverses on $N$ and $H$ respectively, both smooth since $N$ and $H$ are Lie groups. Then the inverse $\iota : G \to G$ on $G$ is given by
    \begin{equation*}
        \iota(g) = (\alpha((\iota_H \circ \pi_H)(g), (\iota_N \circ \pi_N)(g)), (\iota_H\circ \pi_H)(g)),
    \end{equation*}
    which is also a composition of smooth maps and hence itself smooth. Since $G$ is a manifold and the group operations on $G$ are smooth, $G$ is a Lie group.

    \item We first check that $\alpha_g$ is a group automorphism for every $g \in \mathrm{SO}(n)$. Let $g \in \mathrm{SO}(n)$ and $x,y \in \bb{R}^n$. Then $\alpha_g(x+y) = g(x+y) = gx+gy = \alpha_g(x) + \alpha_g(y)$, so $\alpha_g$ is a group homomorphism. Suppose that $\alpha_g(x) = \alpha_g(y)$. Then $gx-gy = 0$, but since $\det g \neq 0$, this implies $x-y = 0$. Therefore, $\alpha_g$ is injective. Moreover, $\alpha_g$ is surjective because $\alpha_g(g^{-1}x) = x$ for all $x \in \bb{R}^n$. Therefore, $\alpha_g$ is an automorphism, so the definition of $E(n)$ as $N \rtimes H$ makes sense.
    
    Since $\bb{R}^n$ and $\mathrm{SO}(n)$ are both Lie groups, and $\alpha$ is obviously smooth, $E(n)$ is a Lie group by (iii). Let $\beta : E(n) \times \bb{R}^n \to \bb{R}^n : ((g,x),y) \mapsto gy+x$ be the natural action. We show now that $\beta$ is free and transitive.

    Let $(g,x) \in E(n)$ and suppose $\beta((g,x),y) = y$ for all $y \in \bb{R}^n$. Then $gy+x = y$, which is equivalent to $x = (\mathbbm{1}-g)(y)$. This can only hold for all $y$ if $g = \mathbbm{1}$ and thus $x=0$, so that $(g,x) = (\mathbbm{1},0)$ is the identity element in $E(n)$. Therefore, $\beta$ is free. Furthermore, let $y,z \in \bb{R}^n$. Then $g((\mathbbm{1}, z-y),y) = y+z-y = z$, so $\beta$ is transitive.

    Since $\beta$ is transitive, its orbit through any $(g,x) \in E(n)$ is $E(n)$ is $E(n)$ itself, so the orbit space is trivial.
\end{enumerate}


\section*{Week 6: Exercise 1} %==============================

\end{document}