\documentclass{article}
\usepackage[a4paper, total={6in, 8in}]{geometry}
\usepackage{amsthm}
\usepackage{amssymb}
\usepackage{amsfonts}
\usepackage{amsmath}
\usepackage{setspace}
\usepackage{xcolor}
\usepackage{cite}
\usepackage[hidelinks]{hyperref}
\usepackage{xurl}
\usepackage{microtype}
\usepackage{enumitem}

% macro for bra kets
\usepackage{mathtools}
\DeclarePairedDelimiter\bra{\langle}{\rvert}
\DeclarePairedDelimiter\ket{\lvert}{\rangle}
\DeclarePairedDelimiterX\braket[2]{\langle}{\rangle}{#1\,\delimsize\vert\,\mathopen{}#2}

\newcommand{\R}{\mathbb{R}}
\newcommand{\C}{\mathbb{C}}
\newcommand{\eps}{\varepsilon}
\renewcommand{\Im}{\mathrm{Im}}
\renewcommand{\Re}{\mathrm{Re}}

\newcommand{\bb}{\mathbb}

\newcommand{\SL}{\mathrm{SL}}
\DeclareMathOperator{\diag}{\mathrm{diag}}
\renewcommand{\d}{\mathrm{d}}

\theoremstyle{remark}
\newtheorem{questionpart}{}
\renewcommand*{\thequestionpart}{\textbf{(\alph{questionpart})}}

\title{Lie Groups -- Assignment 2}
\author{Quinten Vergeylen}
\date{}

\begin{document}
\maketitle
\setstretch{1.1}

\section*{Week 4: Exercise 1} %==============================
\begin{enumerate}[label=(\roman*)]
    \item By Theorem 4.8, $G_e \cong \bb{T}^p \times \bb{R}^q$ for certain $p, q \in \bb{N}$. Therefore, any subgroup of $G_e$ is isomorphic to a group of the form $T_1 \times \cdots \times T_p \times R_1 \times \cdots \times R_q$, where $T_j$ is a subgroup of $\bb{T}$ for every $j$ and $R_k$ is a subgroup of $\bb{R}$ for every $k$. We claim that $\bb{T}^p \cong \bb{T}^p \times \{0\}^p \subset G_e$ is the unique maximal compact subgroup of $G_e$, where $\{0\}$ denotes the trivial subgroup of $\bb{R}$.
    
    It is clear that $\bb{T}^p$ is a compact subgroup of $G_e$. Suppose that $L$ is a subgroup of $G_e$ which contains $\bb{T}^p$. Then $L$ is isomorphic to
    \begin{equation*}
        \bb{T}^p \times R_1 \times \cdots \times R_q
    \end{equation*}
    for certain subgroups $R_k$ of $\bb{R}$. If $R_k$ is any other subgroup of $\bb{R}$ other than $\{0\}$ for any $k$, then $L$ is not compact, since $\{0\}$ is the only compact subgroup of $\bb{R}$. It follows that $\bb{T}^p$ is a maximal compact subgroup of $G_e$. Lastly, we can use the reasoning above to assert that any compact subgroup of $G_e$ is isomorphic to 
    \begin{equation*}
        T_1 \times T_p \times \{0\}^p \cong T_1 \times T_p \subset \bb{T}^p
    \end{equation*}
    for certain subgroups $T_j$ of $\bb{T}$. Therefore, $K = \bb{T}^p$ is the unique maximal compact subgroup of $G_e$.

    \item Let $A = \{ g \in G \mid \exists n \in \bb{Z}_{>0} : g^n \in K \}$. We claim that $A$ is the unique maximal compact subgroup of $G$.
    
    First we show that $A$ is a subgroup of $G$. Let $g, h \in A$ and suppose $g^n \in K$ and $h^m \in K$. Then $(gh)^{nm} = g^{nm}h^{nm} = (g^n)^m (h^m)^n \in K$, where we used that $G$ is abelian and that $K$ is a subgroup of $G$. It follows that $gh \in K$. Furthermore, $(g^{-1})^n = (g^n)^{-1} \in K$, so $g^{-1} \in A$. Therefore, $A$ is a subgroup of $G$.

    Next, we show that $A$ is compact. Let $G_1, \ldots, G_k$ be the connected components of $G$ with the property for each $j \in \{1, \ldots, k\}$ that there exists a $h_j \in G_j$ and an $n_j \in \bb{Z}_{>0}$ such that $h_j^{n_j} \in K$. In other words, $G_1, \ldots, G_k$ are the connected components of $G$ with nonempty intersection with $A$. We know that there are finitely many such $G_j$ since $G$ has finitely many connected components by assumption.
    
    Fix $j \in \{1, \ldots, j\}$ for now and note that the left translation $\ell_{h_j} : G_e \to G_j$ is a diffeomorphism. Indeed, $\ell_{h_j}(e) = h_j \in G_j$ and $\ell_{h_j}(G_e)$ is both open and closed since $G_e$ is both of these and $\ell_{h_j} : G \to G$ is a diffeomorphism. It follows that $\ell_{h_j}(G_e)$ must be the connected component of $G$ containing $h_j$, which is precisely $G_j$. Now pick any $x \in G_j \cap A$, supposing $x^m \in K$. We now have $\ell_{h_j}^{-1}(x) = h_j^{-1}x \in G_e \cong \bb{T}^p \times \bb{R}^q$ and $(h_j^{-1}x)^{mn_j} \in K \cong \bb{T}^p \times \{0\}^p$. Let $\pi : G_e \cong \bb{T}^p \times \bb{R}^q \to \bb{R}^q$ be the projection. The fact that $\pi((h_j^{-1}x)^{mn_j}) = 0 \in \bb{R}^q$ implies that $\pi(h_j^{-1}x) = 0$, since the only real number that becomes zero when added to itself $mn_j$ times is zero. Therefore, $h_j^{-1}x \in K$, and hence $xK = h_jK$. In particular, $x \in h_j K$.

    It now follows that $A = h_1K \cup \cdots \cup h_kK$. Indeed, we have just shown that any element $x$ of $A$ is contained in $h_jK$ if $G_j$ is the connected component of $G$ containing $x$. Conversely, $(h_jg)^{n_j} = h_j^{n_j}g^{n_j} \in K$ for every $g \in K$, so $h_jK \subset A$ for every $j$. Since $A$ is a finite union of compact sets $h_jK$, we have shown that $A$ is compact.
\end{enumerate}


\section*{Week 5: Exercise 3} %==============================



\section*{Week 6: Exercise 1} %==============================

\end{document}