\documentclass{article}
\usepackage[a4paper, total={6in, 8in}]{geometry}
\usepackage{amsthm}
\usepackage{amssymb}
\usepackage{amsfonts}
\usepackage{amsmath}
\usepackage{setspace}
\usepackage{xcolor}
\usepackage{cite}
\usepackage[hidelinks]{hyperref}
\usepackage{xurl}
\usepackage{microtype}
\usepackage{enumitem}

% macro for bra kets
\usepackage{mathtools}
\DeclarePairedDelimiter\bra{\langle}{\rvert}
\DeclarePairedDelimiter\ket{\lvert}{\rangle}
\DeclarePairedDelimiterX\braket[2]{\langle}{\rangle}{#1\,\delimsize\vert\,\mathopen{}#2}

\newcommand{\R}{\mathbb{R}}
\newcommand{\C}{\mathbb{C}}
\newcommand{\eps}{\varepsilon}
\renewcommand{\Im}{\mathrm{Im}}
\renewcommand{\Re}{\mathrm{Re}}
\newcommand{\Hom}{\mathrm{Hom}}
\newcommand{\Aut}{\mathrm{Aut}}

\newcommand{\bb}{\mathbb}

\newcommand{\SL}{\mathrm{SL}}
\DeclareMathOperator{\diag}{\mathrm{diag}}
\renewcommand{\d}{\mathrm{d}}

\theoremstyle{remark}
\newtheorem{questionpart}{}
\renewcommand*{\thequestionpart}{\textbf{(\alph{questionpart})}}

\title{Lie Groups -- Assignment 4}
\author{Quinten Vergeylen}
\date{}

\begin{document}
\maketitle

\setstretch{1.1}

\section*{Week 11, exercise 1}

\begin{enumerate}[label=(\roman*)]
    \item Let $\phi$ be the given map. First we show that $\phi$ is well-defined. Let $T \in \Hom_G(V \oplus V', W)$. We need to show that $T\vert_V : V \to W$ and $T\vert_{V'} : V' \to W$ are linear maps that intertwine the representations. The maps $T\vert_{V}$ and $T\vert_{V'}$ are linear since they are both restrictions of a linear map to a linear subspace. Let $i_V : V \to V \oplus V' : v \mapsto (v, 0)$ be the inclusion and fix $x \in G$. Then we can identify $T\vert_V = T \circ i_V$, so
    \begin{equation*}
        T\vert_V \pi(x)v = T(i_V(\pi(x)v)) = T(\pi(x)v, 0).
    \end{equation*}
    Now we can use the linearity of $\pi'(x)$ to write
    \begin{equation*}
        T(\pi(x)v, 0) = T(\pi(x)v, \pi'(x) 0) = T(\pi \oplus \pi')(x)(v,0) = \rho(x) T(v, 0).
    \end{equation*}
    In the last equality we used the intertwining property of $T$. But
    \begin{equation*}
        T(v, 0) = (T \circ i_V)(v) = T\vert_V v,
    \end{equation*}
    so $T\vert_V \pi(x) = \rho(x) T\vert_V$, which means that $T\vert_V$ is intertwining. A similar argument shows that $T\vert_{V'}$ is also intertwining. Therefore, $\phi$ is well-defined.
    
    Now we show that $\phi$ is linear. Let $T, S \in \Hom_G(V \oplus V', W)$ and $a, b \in \bb{C}$. Then, for all $v \in V$ and $v' \in V$,
    \begin{align*}
        \phi(aT + bS)(v, v') &= (a T + bS)\vert_V \oplus (aT+bS)\vert_{V'}(v,v') \\
        &= (a T\vert_V + bS\vert_V) \oplus (aT\vert_{V'}+bS\vert_{V'})(v,v') \\
        &= (aT\vert_Vv + bS\vert_Vv, \, aT\vert_{V'}v' + bS\vert_{V'}v') \\
        &= a (T\vert_Vv, \, T\vert_{V'}v') + b (S\vert_Vv, \, S\vert_{V'}v') \\
        &= a (T\vert_V \oplus T\vert_{V'})(v, v') + b (S\vert_V \oplus S\vert_{V'})(v, v') \\
        &= (a\phi(T) + b\phi(S))(v, v').
    \end{align*}
    Hence $\phi$ is linear.
    
    Next, we show that $\phi$ is injective. Let $T \in \Hom_G(V \oplus V', W)$ and suppose $\phi(T) = 0$. Then $T\vert_V = 0$ and $T\vert_{V'} = 0$. Therefore, for all $(v,v') \in V \oplus V'$,
    \begin{equation*}
        T(v, v') = T(v, 0) + T(0, v') = T\vert_V v + T\vert_{V'}v'= 0
    \end{equation*}
    and thus $T = 0$. By the linearity of $\phi$, it follows that $\phi$ is injective.
    
    Lastly, we show that $\phi$ is surjective. It suffices to check that if $A \in \Hom_G(V, W)$ and $B \in \Hom_G(V',W)$, then the map
    \begin{equation*}
        T : V \oplus V' \to W : (v, v') \mapsto Av + Bv'
    \end{equation*}
    lies in $\Hom_G(V\oplus V', W)$, since $T\vert_V = A$ and $T|\vert_{V'} = B$. First we show that $T$ is linear: for all $a, b \in \bb{C}$, $(v,v'), (w,w') \in V \oplus V'$, we have
    \begin{equation*}
        T(a(v,v') + b(w,w')) = T(av + bw, av' + bw') = A(av + bw) + B(av' + bw').
    \end{equation*}
    Using the linearity of $A$ and $B$, this becomes
    \begin{equation*}
        aAv + bAw + aBv' + bBw' = aT(v,v') + bT(w,w').
    \end{equation*}
    hence $T$ is linear. We only need to show that $T$ intertwines the representations. Let $x \in G$ and $(v,v') \in V\oplus V'$. Then
    \begin{equation*}
        T(\pi \oplus \pi')(x)(v,v') = T(\pi(x)v, \pi'(x)v') = A\pi(x)v + B\pi'(x)v'.
    \end{equation*}
    Using the intertwining property of $A$ and $B$, this becomes
    \begin{equation*}
        \rho(x)Av + \rho(x)Bv' = \rho(x)T(v,v'),
    \end{equation*}
    so $T$ is intertwining. This concludes the proof that $\phi$ is a linear bijection.

    \item Suppose
    \begin{equation*}
        (\pi, V) = (\pi_1 \oplus \cdots \oplus \pi_r, V_1 \oplus \cdots V_r) \quad \text{and} \quad (\rho, W) = (\rho_1 \oplus \cdots \oplus \rho_s, W_1 \oplus \cdots \oplus W_s)
    \end{equation*}
    are decompositions of $(\pi, V)$ and $(\rho, W)$ into irreducible representations $(\pi_j, V_j)$ and $(\rho_i, W_i)$, for $j \in \{1, \ldots, r\}$ and $i \in \{1, \ldots, s\}$. These exist by Corollary 20.19 (van de Ban), where $\pi$ and $\rho$ are unitarisable since $G$ is assumed compact.

    Let $p_i : W \to W_i : (w_1, \ldots, w_s) \mapsto w_i$ be the projection for each $i$. Fix $j \in \{1, \ldots, r\}$. Clearly, the map
    \begin{equation*}
        \psi : \Hom_G(V_j, W_1 \oplus \cdots \oplus W_s) \cong \Hom_G(V_j, W) \to \bigoplus_{i=1}^s \Hom_G(V_j, W_i) : T \mapsto \bigoplus_{i=1}^s (p_i \circ T)
    \end{equation*}
    is a linear isomorphism. It is well-defined since, for every $i \in \{1, \ldots, s\}$, $x \in G$ and $v \in V_j$, we have
    \begin{equation*}
        (p_i \circ T) \pi_j(x) v = p_i(T\pi_j(x)v) = p_i(\rho(x)Tv) = \rho_i(x) (p_i \circ T)(v).
    \end{equation*}
    It follows that
    \begin{equation*}
        \dim_\bb{C} \Hom_G(V_j, W_1 \oplus \cdots \oplus W_s) = \sum_{i=1}^{s} \dim_\bb{C} \Hom_G(V_j, W_i).
    \end{equation*}

    By combining this equation with part (i), we have
    \begin{equation*}
        \dim_\bb{C} \Hom_G(V, W) = \sum_{i=1}^{s} \sum_{j=1}^{r} \dim_\bb{C} \Hom_G(V_j, W_i).
    \end{equation*}
    Meanwhile, Lemma 22.7 says that
    \begin{equation*}
        \langle \chi_\pi, \chi_\rho \rangle = \sum_{i=1}^{s} \sum_{j=1}^{r} \langle \chi_{\pi_j}, \chi_{\rho_i} \rangle.
    \end{equation*}
    It therefore suffices to show that $\langle \chi_{\pi_j}, \chi_{\rho_i} \rangle = \dim_\bb{C} \Hom_G(V_j, W_i)$ for all $j$ and $i$.

    By Lemma 22.10, $\langle \chi_{\pi_j}, \chi_{\rho_i} \rangle = 1$ if $\pi_j \sim \rho_i$ and $\langle \chi_{\pi_j}, \chi_{\rho_i} \rangle = 0$ if $\pi_j \not\sim \rho_i$. Suppose $\pi_j \sim \rho_i$. Then in particular we can identify $V_j$ and $W_i$ as linear spaces, so $\Hom_G(V_j, W_i) \cong \mathrm{End}_G(V_j)$. But the latter is one-dimensional by Schur's lemma, so $\dim_\bb{C} \Hom_G(V_j, W_i) = 1 = \langle \chi_{\pi_j}, \chi_{\rho_i} \rangle$. If $\pi_j \not\sim \rho_i$, then every intertwiner $V_j \to W_i$ is trivial (Lemma 20.30), so $\dim_\bb{C} \Hom_G(V_j, W_i) = 0 = \langle \chi_{\pi_j}, \chi_{\rho_i} \rangle$. This completes the proof.
\end{enumerate}

\section*{Week 12, exercise 1}
\begin{enumerate}[label=(\roman*)]
    \item I will use the content of Section 26 in van de Ban's notes. Since $G$ is finite, there are finitely many conjugacy classes, say $C_1, \ldots, C_r$. For each $j \in \{1, \ldots, r\}$, define
    \begin{equation*}
        f_j : G \to \bb{C} : x \mapsto \begin{cases}
            1 & \text{if $x \in C_j$,} \\
            0 & \text{otherwise.}
        \end{cases}
    \end{equation*}
    Then $f_1, \ldots, f_r$ clearly form a basis for $L^2(G, \text{class})$, the space of square integrable class functions on $G$. In other words, $\dim L^2(G, \text{class}) = r$. But by Lemma 26.1, $\dim L^2(G, \text{class}) = \#\hat{G}$. Therefore, $\hat{G}$ is finite.

    \item By Lemma 22.11, we can identify $\pi$ with $\bigoplus_{\delta \in \hat{G}} m(\delta, \pi) \delta$.
    For every $\delta \in \hat{G}$, let $V_\delta$ be its representation space. Then we can identify $V_\pi$ with $\bigoplus_{\delta \in \hat{G}} V_\delta$,
    where the $V_\delta$ are orthogonal subspaces, according to Corollary 20.19 and Lemma 20.18. For every $\delta \in \hat{G}$, let $p_\delta : V_\pi \to V_\delta$ be the projection according to this decomposition. Now we have
    \begin{equation*}
        \langle \pi(g) v_1, v_2 \rangle = \langle ( \bigoplus_{\delta \in \hat{G}} m(\delta, \pi) \delta(g) ) v_1, v_2 \rangle = \sum_{\delta \in \hat{G}} m(\delta, \pi) \langle  \delta(g) p_\delta(v_1), p_\delta(v_2) \rangle_\delta
    \end{equation*}
    for all $g \in G$ by the orthogonality of the irreducible spaces $V_\delta$, where $\langle \cdot, \cdot \rangle_\delta$ is the respective induced Hermitian inner product. But $g \mapsto \langle  \delta(g) p_\delta(v_1), p_\delta(v_2) \rangle_\delta$ belongs to $C(G)_\delta$ since it is a matrix coefficient of $\delta$. Therefore, $g \mapsto \langle \pi(g) v_1, v_2 \rangle$ belongs to $\mathcal{R}(G)$.

    \item Let $\phi \in \mathcal{R}(G)$. Then there exists a $\phi_\delta \in C(G)_\delta$ for each $\delta \in \hat{G}$ such that $\phi = \sum_{\delta \in \hat{G}} \phi_\delta$. In turn, for each $\delta \in \hat{G}$, there exist matrix elements $m^\delta_1, \ldots, m^\delta_{n_\delta}$ of $\delta$ such that $\phi_\delta = \sum_{j=1}^{n_\delta} m^\delta_j$. Fix $\delta \in \hat{G}$ and $j \in \{1, \ldots, n_\delta\}$, and suppose that $v, w \in V_\delta$ are such that $m^\delta_j(g) = \langle \delta(g) v, w \rangle$ for all $g \in G$. Then, for all $x, g \in G$, we have $m^\delta_j(gx^{-1}) = \langle \delta(gx^{-1}) v, w \rangle = \langle \delta(g)\delta(x^{-1}) v, w \rangle$. Since $\delta(x^{-1})v \in V_\delta$, the map $g \mapsto m^\delta_j(gx^{-1})$ is a matrix element for all $x \in G$. Similarly, $g \mapsto m^\delta_j(xg)$ is a matrix element for all $x \in G$. It follows that $g \mapsto \phi_\delta(gx^{-1})$ and $g \mapsto \phi_\delta(xg)$ belong to $C(G)_\delta$ for all $x \in G$, and therefore $g \mapsto \phi(gx^{-1})$ and $g \mapsto \phi(xg)$ belong to $\mathcal{R}(G)$ for all $x \in G$.
    
    Now let $f \in \mathcal{R}(G)^\perp$. For any $\phi \in \mathcal{R}(G)$ and $x \in G$, we have
    \begin{equation} \label{eq:Rxf-phi-product}
        \langle R_x(f), \phi \rangle = |G|^{-1}\sum_{g \in G} f(gx) \overline{\phi(g)} = |G|^{-1}\sum_{g \in G} f(g) \overline{\phi(gx^{-1})}.
    \end{equation}
    We just showed that $\psi_x : G \to \bb{C} : g \mapsto \phi(gx^{-1})$ belongs to $\mathcal{R}(G)$, so \eqref{eq:Rxf-phi-product} becomes $\langle f, \psi_x \rangle = 0$. Since $\phi$ was arbitrary, $R_x(f) \in \mathcal{R}(G)^\perp$. But $x$ was also arbitrary, so $\mathcal{R}(G)^\perp$ is invariant for the right regular representation of $G$. Analogouosly, $\mathcal{R}(G)^\perp$ is invariant for the left regular representation of $G$.

    \item The finiteness of $G$ implies that $\mathcal{R}(G)^\perp$ is finite-dimensional: every function on $G$ is determined by finitely many complex numbers, namely its value at each of the elements of $G$. Therefore, and by (iii), $\mathcal{R}(G)^\perp$ is a finite-dimensional right $G$-invariant subspace of $C(G)$. Let $f \in \mathcal{R}(G)^\perp$ and let $f_1, \ldots, f_n$ be an orthonormal basis for $\mathcal{R}(G)^\perp$ with respect to an inner product on $\mathcal{R}(G)^\perp$ which makes $R$ unitary. Then, for every $g,x \in G$ we have
    \begin{equation*}
        R_x(f)(g) = f(gx) = \sum_{j=1}^{n} \langle R_x(f), f_j \rangle f_j(g).
    \end{equation*}
    Setting $g = e$, this gives
    \begin{equation*}
        f(x) = \sum_{j=1}^{n} \langle R_x(f), f_j \rangle f_j(e).
    \end{equation*}
    But by (ii), $x \mapsto \langle R_x(f), f_j \rangle$ belongs to $\mathcal{R}(G)$ for all $j$. Therefore, $f \in \mathcal{R}(G)$ and thus $f = 0$. This holds for any $f \in \mathcal{R}(G)^\perp$, so $\mathcal{R}(G)^\perp = \{0\}$. It follows that $C(G) = \mathcal{R}(G) \oplus \mathcal{R}(G)^\perp = \mathcal{R}(G)$.
\end{enumerate}

\section*{Week 13, exercise 2}
\begin{enumerate}[label=(\roman*)]
    \item Let $\phi, \psi \in \Aut(G)$. Then $D(\phi \circ \psi) = T_{\psi(e)} \phi \circ T_e \psi$ by the chain rule. But $\psi$ is an automorphism, so $\psi(e) = e$. Therefore, $D(\phi \circ \psi) = T_e \phi \circ T_e \psi = D(\phi) \circ D(\psi)$, so $D$ is a group homomorphism.
    
    Now suppose $D(\phi) = \mathrm{id}_\mathfrak{g}$, the identity on $\mathfrak{g}$. Let $g \in G$. Since $G$ is connected, there exist $X_1, \ldots, X_k \in \mathfrak{g}$ such that $g = \exp(X_1) \cdots \exp(X_k)$ (Proposition 3.15 in Koelink's notes). But then
    \begin{align*}
        \phi(g) &= \phi(\exp X_1) \cdots \phi(\exp X_k) &\text{($\phi$ is a group homomorphism)} \\
        &= \exp(T_e \phi (X_1)) \cdots \exp(T_e \phi (X_k)) &\text{(Koelink, Proposition 2.10)} \\
        &= \exp(X_1) \cdots \exp(X_k) &\text{($T_e \phi = \mathrm{id}_\mathfrak{g}$)} \\
        &= g.
    \end{align*}
    Therefore, $\phi = \mathrm{id_G}$. Since $D$ is a group homomorphism, this implies that $D$ is injective.

    \item Let $O \subset \mathfrak{g}$ be an open neighbourhood of $0$ such that $\exp : O \to \exp(O)$ is a diffeomorphism. Since $f_n(\exp X) = \exp(T_e f_n (X)) \to \exp(\psi(X))$ by the continuity of $\exp$, for all $X \in O$, the sequence $(f_n)_n$ converges pointwise on $\exp(O)$. Now fix $g \in G$, not necessarily in $\exp(O)$. Then there exist $X_1, \ldots, X_k \in \mathfrak{g}$ such that $g = \exp(X_1) \cdots \exp(X_k)$. Let $m \in \bb{N}$ be large enough that $X_j/m \in O$ for all $j \in \{1, \ldots, k\}$. Write $Y_j = X_j/m$ for all $j$. Then, by Proposition 2.9 (Koelink),
    \begin{equation*}
        g = \exp(mY_1) \cdots \exp(mY_k) = (\exp Y_1)^m \cdots (\exp Y_k)^m.
    \end{equation*}
    Now we see that
    \begin{equation} \label{eq:limit-of-fn}
        f_n(g) = f_n(\exp Y_1)^m \cdots f_n(\exp Y_k)^m \to (\exp \psi(Y_1))^m \cdots (\exp \psi(Y_k))^m
    \end{equation}
    by the homomorphism property of $f$ and the continuity of $\exp$ and the group multiplication. Therefore, there exists a pointwise limit $f$ of $(f_n)_n$.

    We now show that $f$ is an automorphism of $G$. Let $g, h \in G$. Then
    \begin{equation*}
        f(gh) = \lim_{n \to \infty} f_n(gh) = \lim_{n \to \infty} f_n(g)f_n(h) = f(g)f(h)
    \end{equation*}
    by the continuity of multiplication, so $f$ is a group homomorphism. Let $F : G \to G$ be the function that takes $g = (\exp Y_1)^m \cdots (\exp Y_k)^m$ to
    \begin{equation*}
        F(g) = (\exp (-\psi(Y_k)))^m \cdots (\exp (-\psi(Y_1)))^m.
    \end{equation*}
    By Proposition 2.9, $f \circ F = F \circ f = \mathrm{id}_G$, so $f$ is bijective. Note that $F$ is well-defined: if
    \begin{equation*}
        g = (\exp Y_1)^m \cdots (\exp Y_k)^m = (\exp Z_1)^p \cdots (\exp Z_l)^p
    \end{equation*}
    for $Y_1, \ldots, Y_k, Z_1, \ldots, Z_l \in O$, and $m,p \in \bb{N}$, we know by the uniqueness of the limit \eqref{eq:limit-of-fn} that
    \begin{equation*}
        (\exp \psi(Y_1))^m \cdots (\exp \psi(Y_k))^m = (\exp \psi(Z_1))^p \cdots (\exp \psi(Z_l))^p
    \end{equation*}
    and hence by taking the inverse,
    \begin{equation*}
        (\exp (- \psi(Y_k)))^m \cdots (\exp (- \psi(Y_1)))^m = (\exp (-\psi(Z_l)))^p \cdots (\exp (-\psi(Z_1)))^p.
    \end{equation*}
    
    % Next we show that $f$ is injective. Note first that
    
    % Suppose $f$ is not injective. Then there is a $g \in \ker f \setminus\{e\}$. In particular, since $f(g) = \lim_{n \to \infty} f_n(g) = e$, and 
    % there is an $n_0 \in \bb{N}$ such that $f_n(g) \in \exp(\psi(O))$ for all $n \geq n_0$.
    
    \item \textcolor{red}{TO DO: $f$ continuous}

    By Proposition 5.1 (Koelink), $f$ is smooth.

    \item 
\end{enumerate}
\end{document}